%% Adaptado de
%% http://www.ctan.org/tex-archive/macros/latex/contrib/IEEEtran/
%% Traduzido para o congresso de IC da USP
%%*****************************************************************************
% Não modificar

\documentclass[twoside,conference,a4paper]{IEEEtran}

%******************************************************************************
% Não modificar
\usepackage{IEEEtsup} % Definições complementares e modificações.
\usepackage[utf8]{inputenc} % Disponibiliza acentos.
\usepackage[english,brazilian]{babel}
%% Disponibiliza Inglês e Português do Brasil.
\usepackage{latexsym,amsfonts,amssymb} % Disponibiliza fontes adicionais.
\usepackage{theorem}
\usepackage[cmex10]{amsmath} % Pacote matemático básico
\usepackage{url}
%\usepackage[portuges,brazil,english]{babel}
\usepackage{graphicx}
\usepackage{amsmath}
\usepackage{amssymb}
\usepackage{color}
\usepackage{float}
\usepackage[pagebackref=true,breaklinks=true,letterpaper=true,colorlinks,bookmarks=false]{hyperref}
\usepackage[tight,footnotesize]{subfigure}
\usepackage[noadjust]{cite} % Disponibiliza melhorias em citações.
%%*****************************************************************************

\begin{document}
\selectlanguage{brazilian}
\renewcommand{\IEEEkeywordsname}{Palavras-chave}

%%*****************************************************************************

\urlstyle{tt}
% Indicar o nome do autor e o curso/nível (grad-mestrado-doutorado-especial)
\title{Técnicas de Inteligência Artificial para fazer robo NAO aprender a andar}
\author{%
 \IEEEauthorblockN{Jucélio Fonseca\IEEEauthorrefmark{1}, Lucas Padilha\IEEEauthorrefmark{1}, Luciano Sabença\IEEEauthorrefmark{1}, Tomás Silva Queiroga\IEEEauthorrefmark{1}}
 \IEEEauthorblockA{\IEEEauthorrefmark{1}%
                   Ciência da Computação - Graduação}
}

%%*****************************************************************************

\maketitle

%%*****************************************************************************
% Resumo do trabalho
\begin{abstract}
Este trabalho teve como objetivo mostrar diferentes abordagens de inteligência articial (IA) para fazer um robo bípede (NAO) andar utilizando o simulador VREP. Fez-se uso de redes neurais artificiais (RNA), aprendizado por reforço e algortimo genéticos. Estudaremos os prós e contras de cada técnica, além dos resultados obtidos. 
\end{abstract}

% Indique três palavras-chave que descrevem o trabalho
\begin{IEEEkeywords}
 Robótica, NAO, VREP, aprendizado por reforço, aprendizado supervisionado, algoritmo genético
\end{IEEEkeywords}

%%*****************************************************************************
% Modifique as seções de acordo com o seu projeto

\section{Introdução}

Uma das coisas mais excepcionais da natureza é a capacidade de andar com somente dois apoios. A dificuldade desse método é tão alta que pouquissímos animais conseguiram evoluir para atingir isso, sendo o ser-humano, obviamente, o exemplo mais acabado de tal habilidade. Apesar a dificuldade, há diversas vantagens nesse tipo de movimento, como, por exemplo: economia de energia, a capacidade de atingir distâncias maiores, além de, no caso dos humanos, amplicar a capacidade de visão. 

Em robótica, há muito deseja-se conseguir que robôs bipedes andem de forma autônoma e eficiente. Porém o desafio é imenso e ainda não foi totalmente resolvido. Sendo assim, estudaremos este campo buscando fugir das abordagens tradicionais, baseadas em modelos precisos e métodos matemáticos, para explorar a métodos baseados em inteligência artificial (IA). 
Neste trabalho, desenvolvemos diferentes modelos e implementações para o problema de locomoção bípede num robô NAO. O robô é um modelo humanóide da empresa francesa Aldebaran extremamente complexo, com 26 graus-de-liberdade (juntas) e diversos tipos de sensores.  Para testar, usaremos o simulador de robótica \textsl{VREP}, disponibilizado pela empresa %TODO%. 

%% TODO colocar figura do Robô

O resto do trabalho está dividido da seguinte maneira: na seção \ref{dificuldades} trataremos das dificuldades do problema de locomoção bípede, mostraremos as soluções e modelos normalmente adotados para resolvê-lo e também mostraremos brevemente o estado-da-arte para o robô NAO.
Na seção \ref{aprendizagem_por_reforco}, mostraremos o modelo que montamos usando a técnica de aprendizagem por reforço, seus resultados e dificuldades. Na seção \ref{aprendizagem_supervisionada} trataremos do modelo que implementamos baseado em aprendizagem supervisionada e redes-neurais, mostraremos os resultados e evolução ao longo do tempo, além de também destacar os problemas e limitações dessa abordagem. Faremos o mesmo das seções anteriores para a seção \ref{algoritmo_geneticos}. Por fim, daremos um panôrama do problemae e dos modelos que usamos nesse trabalho, também discutiremos os resultados que obtemos e propôremos soluções para trabalhos futuros nesse tema na seção \ref{conclusoes}.



 \section{Dificuldades do Problema} \label{dificuldades}

%% Tratar das dificuldades da locomoçao, centro de massa, modelos matematicos, ZMP, etc
%% Superficialmente

 \section{Aprendizagem Por Reforço} \label{aprendizagem_por_reforco}
 

 \section{Aprendizagem Supervisionada} \label{aprendizagem_por_reforco}
 
  \section{Algoritmos Genéticos} \label{algoritmos_geneticos}
  
   \section{Conclusões} \label{conclusoes}
 
\section{Métodos}

Para ser alcançado os resultados desejados foram executados os seguintes passos:
\begin{enumerate}
 \item
\end{enumerate}


\section{Resultados e Discussão}

\section{Conclusões}

%******************************************************************************
% Referências - Definidas no arquivo Relatorio.bib
\nocite{Aula}
\bibliographystyle{IEEEtran}

\bibliography{Relatorio}

\end{document}
