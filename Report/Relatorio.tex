
%% Adaptado de
%% http://www.ctan.org/tex-archive/macros/latex/contrib/IEEEtran/
%% Traduzido para o congresso de IC da USP
%%*****************************************************************************
% Não modificar

\documentclass[twoside,conference,a4paper]{IEEEtran}

%******************************************************************************
% Não modificar
\usepackage{IEEEtsup} % Definições complementares e modificações.
\usepackage[utf8]{inputenc} % Disponibiliza acentos.
\usepackage[english,brazilian]{babel}
%% Disponibiliza Inglês e Português do Brasil.
\usepackage{latexsym,amsfonts,amssymb} % Disponibiliza fontes adicionais.
\usepackage{theorem}
\usepackage[cmex10]{amsmath} % Pacote matemático básico
\usepackage{url}
%\usepackage[portuges,brazil,english]{babel}
\usepackage{graphicx}
\usepackage{amsmath}
\usepackage{amssymb}
\usepackage{color}
\usepackage{float}
\usepackage[pagebackref=true,breaklinks=true,letterpaper=true,colorlinks,bookmarks=false]{hyperref}
\usepackage[tight,footnotesize]{subfigure}
\usepackage[noadjust]{cite} % Disponibiliza melhorias em citações.
%%*****************************************************************************

\begin{document}
\selectlanguage{brazilian}
\renewcommand{\IEEEkeywordsname}{Palavras-chave}

%%*****************************************************************************

\urlstyle{tt}
% Indicar o nome do autor e o curso/nível (grad-mestrado-doutorado-especial)
\title{Técnicas de Inteligência Artificial para fazer robo NAO aprender a andar}
\author{%
 \IEEEauthorblockN{Jucélio Fonseca\IEEEauthorrefmark{1}, Lucas Padilha\IEEEauthorrefmark{1}, Luciano Sabença\IEEEauthorrefmark{1}, Tomás Silva Queiroga\IEEEauthorrefmark{1}}
 \IEEEauthorblockA{\IEEEauthorrefmark{1}%
                   Ciência da Computação - Graduação}
}

%%*****************************************************************************

\maketitle

%%*****************************************************************************
% Resumo do trabalho
\begin{abstract}
O objetivo deste trabalho é mostrar diferentes abordagens de inteligência articial (IA) para fazer um robo bípede (NAO) andar utilizando o simulador VREP. Fez-se uso de redes neurais artificiais (RNA), aprendizado por reforço e algortimo genéticos. Estudaremos os prós e contras de cada técnica, e exibr dos resultados obtidos.
\end{abstract}

% Indique três palavras-chave que descrevem o trabalho
\begin{IEEEkeywords}
 Robótica, NAO, VREP, aprendizado por reforço, aprendizado supervisionado, algoritmo genético
\end{IEEEkeywords}

%%*****************************************************************************
% Modifique as seções de acordo com o seu projeto

\section{Introdução}

Uma das coisas mais excepcionais da natureza é a capacidade de andar com somente dois apoios. A dificuldade desse método é tão alta que pouquissímos animais conseguiram evoluir para atingir isso, sendo o ser-humano, obviamente, um bom exemplo de tal habilidade. Apesar da dificuldade, há diversas vantagens nesse tipo de movimento, como, por exemplo, a economia de energia, a capacidade de atingir distâncias maiores, além de, no caso dos humanos, amplicar a capacidade de visão.

Em robótica, há muito deseja-se conseguir que robôs bipedes andem de forma autônoma, eficiente e resiliente. Porém o desafio é imenso e ainda não foi totalmente resolvido. Sendo assim, estudaremos este campo buscando fugir das abordagens tradicionais, baseadas em modelos precisos e métodos matemáticos e físicos, para explorar a métodos baseados em inteligência artificial (IA).
Neste trabalho, desenvolvemos diferentes modelos e implementações para o problema de locomoção bípede num robô NAO (Figura \ref{fig:fig1}).  Para testar, usaremos o simulador de robótica \textsl{VREP}, disponibilizado pela empresa Coppelia Robotics GmbH.

\begin{figure}[H]
 \centering
 \includegraphics[width=1\hsize]{figuras/{NAO}.png}
 \caption{Exemplo robô NAO utilizado neste trabalho, um modelo humanóide da empresa francesa Aldebaran extremamente complexo, com 26 graus-de-liberdade (juntas) e diversos tipos de sensores}
 \label{fig:fig1}
\end{figure}

O resto do trabalho está dividido da seguinte maneira: na seção \ref{dificuldades} trataremos das dificuldades do problema de locomoção bípede, mostraremos as soluções e modelos normalmente adotados para resolvê-lo e também mostraremos brevemente o estado-da-arte para o robô NAO.

Na seção \ref{aprendizagem_por_reforco}, mostraremos o modelo que montamos usando a técnica de aprendizagem por reforço, seus resultados e dificuldades. Na seção \ref{aprendizagem_supervisionada} trataremos do modelo que implementamos baseado em aprendizagem supervisionada e redes-neurais artificiais, mostraremos os resultados e evolução ao longo do tempo, além de também destacar os problemas e limitações dessa abordagem. Faremos o mesmo das seções anteriores para a seção \ref{algoritmos_geneticos}. Por fim, daremos um panôrama do problema e dos modelos que usamos nesse trabalho, também discutiremos os resultados que obtemos e propôremos soluções para trabalhos futuros nesse tema na seção \ref{conclusoes}.

\section{Dificuldades do Problema} \label{dificuldades}
A habilidade de caminhar é algo tão automático para seres humanos quanto complexo. Sob um primeiro olhar, pode parecer simplesmente um movimento de pernas coordenados, porém, ao observar mais minunsiosamente, percebe-se que vai muito além disso, até porque, apenas para realizar tal movimento de pernas precisa-se fazer uso do quadril, joelho, tornozelo e pés, rotacionando-os cada um em uma maneira específica em mais de um eixo. Nota-se também que para caminhadas mais rápidas, é necessário também mexer a coluna, ombros, cotovelos, pulso, etc. e que para qualquer caminhada, utilizar de tais partes também é fundamental para manter o equilíbrio, mesmo que suavemente.

O problema de andar então trata-se de realizar uma combinação em função do tempo de qual ângulo qual parte deve assumir para que seja possível sair do lugar. Enumerando rapidamente, observa-se que necessita-se saber, para cada intervalo de tempo, o ângulo do pé, tornozelo, joelho, quadril, coluna, ombros, cotovelo, pulsos, pescoço, ao longo do eixo x, y e z, totalizando em 48 incógnitas que se influenciam. É óbvio então que trata-se de um sistema complexo, sem soluções triviais, que por sua vez é estudado há vários anos.

Uma das abordagens analíticas mais comuns de se resolver o sistema é considerando o centro de massa do ser bípede que se deseja fazer andar. Neste modelo, o centro de massa deve permanecer sempre próximo do seu estado original, variando apenas, claro, ao longo do eixo em que se deseja caminhar através. De maneira simplificada, essa forma de tratar o problema se assemelha ao problema de controlar um pêndulo invertido acoplado a um objeto móvel perpenticular a sua trajetória.

Ponto de momento zero (ZMP - Zero Moment Point) é um dos métodos utilizados para resolver o problema considerando o centro de massa. Ele consiste, basicamente, em encontrar o ponto em que o movimento gera momento horizontal zero no centro de massa, ou seja, um equilíbrio, uma estabilidade dinâmica, visto que caminhar uma órbita periódica com uma fase estável alternada com um instável.

Não importa qual for a abordagem utilizando modelos matemáticos e física, até o momento todas elas tem algo nível de complexidade de resolução e computação, e muitas vezes as soluções tem difícil adaptação para outros robôs.

\section{Aprendizagem Por Reforço} \label{aprendizagem_por_reforco}

\section{Aprendizagem Supervisionada} \label{aprendizagem_supervisionada}

\section{Algoritmos Genéticos} \label{algoritmos_geneticos}
Dada a proximidade que os algoritmos genéticos buscam ter com a natureza, e que os seres humanos são frutos de inúmeras evoluções de populações distintas de animais, parece natural avançar no problema de andar utilizando esta técnica da inteligência artificial. Nela buscamos encontrar, depois de várias gerações, um robô capaz de andar, sem cair, indeterminadamente.

Para isso acontecer, antes de definir tamanho ideal para população, taxa de mutação, pontos de crossover, é necessário formular uma boa função fitness, e codificar um bom cromossomo.


Cada indivíduo (cromossomo) da população é um vetor com $6N$ posições, onde $N$ é o número de juntas em que queremos otimizar suas rotações em função do tempo, e o número $6$ se dá por ser o número de constantes estipuladas em cada função de rotação em função do tempo $f(t) = \sum_{i=1}^6 a_i sen(b_i t)$. Desta forma cada articulação poderá ser utilizada de forma independente em perídos diferentes serapadas das demais.

A função fitness inicial que utilizamos foi bem simples: distância percorrida até o robô cair, sendo que cair é: centro de massa do robô ter girado mais de 90º em qualquer eixo, e parar de se deslocar por alguns instantes. Para isso, para cada indivíduo da população, executou-se uma simulação, utilizando os valores decodificados de seu cromossomo como as constantes da função de rotação citada. A cada intervalo de 0.01s todas as articulações tem seus angulos atualizados de acordo com sua respectiva função, e é verificado se o robo andou, caiu, ficou parado.



\section{Conclusões} \label{conclusoes}

%******************************************************************************
% Referências - Definidas no arquivo Relatorio.bib
\nocite{Aula}
\bibliographystyle{IEEEtran}

\bibliography{Relatorio}

\end{document}
