%% Adaptado de
%% http://www.ctan.org/tex-archive/macros/latex/contrib/IEEEtran/
%% Traduzido para o congresso de IC da USP
%%*****************************************************************************
% Não modificar

\documentclass[twoside,conference,a4paper]{IEEEtran}

%******************************************************************************
% Não modificar
\usepackage{IEEEtsup} % Definições complementares e modificações.
\usepackage[utf8]{inputenc} % Disponibiliza acentos.
\usepackage[english,brazilian]{babel}
%% Disponibiliza Inglês e Português do Brasil.
\usepackage{latexsym,amsfonts,amssymb} % Disponibiliza fontes adicionais.
\usepackage{theorem}
\usepackage[cmex10]{amsmath} % Pacote matemático básico
\usepackage{url}
%\usepackage[portuges,brazil,english]{babel}
\usepackage{graphicx}
\usepackage{amsmath}
\usepackage{amssymb}
\usepackage{color}
\usepackage{float}
\usepackage[pagebackref=true,breaklinks=true,letterpaper=true,colorlinks,bookmarks=false]{hyperref}
\usepackage[tight,footnotesize]{subfigure}
\usepackage[noadjust]{cite} % Disponibiliza melhorias em citações.
%%*****************************************************************************

\begin{document}
\selectlanguage{brazilian}
\renewcommand{\IEEEkeywordsname}{Palavras-chave}

%%*****************************************************************************

\urlstyle{tt}
% Indicar o nome do autor e o curso/nível (grad-mestrado-doutorado-especial)
\title{Técnicas de Inteligência Artificial para fazer robo NAO aprender a andar}
\author{%
 \IEEEauthorblockN{Jucélio Fonseca\IEEEauthorrefmark{1}, Lucas Padilha\IEEEauthorrefmark{1}, Luciano Sabença\IEEEauthorrefmark{1}, Tomás Silva Queiroga\IEEEauthorrefmark{1}}
 \IEEEauthorblockA{\IEEEauthorrefmark{1}%
                   Ciência da Computação - Graduação}
}

%%*****************************************************************************

\maketitle

%%*****************************************************************************
% Resumo do trabalho
\begin{abstract}
Este trabalho teve como objetivo mostrar .
\end{abstract}

% Indique três palavras-chave que descrevem o trabalho
\begin{IEEEkeywords}
 Robótica, NAO, VREP, aprendizado por reforço, aprendizado monitorado, algoritmo genético
\end{IEEEkeywords}

%%*****************************************************************************
% Modifique as seções de acordo com o seu projeto

\section{Introdução}

Em robótica
\section{Trabalho Proposto}

O presente trabalho se propõe a implementar em Python
\section{Métodos}

Para ser alcançado os resultados desejados foram executados os seguintes passos:
\begin{enumerate}
 \item
\end{enumerate}


\section{Resultados e Discussão}

\section{Conclusões}

%******************************************************************************
% Referências - Definidas no arquivo Relatorio.bib
\nocite{Aula}
\bibliographystyle{IEEEtran}

\bibliography{Relatorio}

\end{document}
